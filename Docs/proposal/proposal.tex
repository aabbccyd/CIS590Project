\documentclass[12pt]{article}
\usepackage[margin=1in]{geometry}
\usepackage{setspace}
\bibliographystyle{ieeetr}  
\usepackage{titling}
%\bibliographystyle{plain}

\setlength{\droptitle}{-2cm}
\title{CIS 590 Project Proposal}
\author{Kewen Meng, Hannah Pruse and Yiding Wang}
\date{\today}
\begin{document}
\maketitle
\vspace{-10mm}
\section*{Proposed Topic}
We wish to explore the issues surrounding the public policy of violent video games. This area is broad, with many issues arising regarding free speech, regulation, and censorship. 

\section*{Background}
\indent\indent Since the advent of video games, there has been a controversy surrounding the games and their content. Violence and gore are very prevalent in many of the most popular titles and, indeed, they are often a selling point for developers. Games are thought to be more influential than films, as the player essentially becomes the character and guides their actions. Again echoing the sentiment expressed in ``Digital Nation," this technology is moving so fast that we do not know its longterm effects that is has on us and our minds, especially the developing minds of children.

Recent increases in gun-related violence initiated by young people has caused quite a stir and has lead people to \textbf{suspect if} there \textbf{are some connections} between such acts of violence and video game play. \textbf{Varieties of} studies into this have shown correlations between violent game play and hostile behaviors, violent responses, and even reduced school performance \cite{barlett2009, gentile2004, anderson2003}. \textbf{Admittedly, for now the causal relationship between violent video games and aggressive behaviors still wait to be proved and further established due to the lack of supports of abundant experiments and theories analysis. However, it is not hard to see the influences of these video games on our new generations depending on their degraded communication skills, morality, health, and even their own personality. Consequently, people can easily find these correlations, either directly or indirectly, between their hostile acts and these video games.} As a result, many governmental policies have arisen in an attempt to restrict access of video games with mature content to underage individuals. Censorship is another measure taken to protect youth, and in some countries censorship is not a voluntary act.

Purchase restrictions and censorship are not a complete answer, however, as game developers feel their right to free speech is not being respected if their work is being altered or restricted in any way. Social norms of modern society, in which youth are gaining increased irreverence for law surrounding technology renders many laws useless. Norms further complicate the issue, as generally parents and guardians of children are not as savvy with technology and modern culture, and therefore are unaware of the negative content present in the games their children play. Lastly, enforcement of age limits on software is incredibly difficult. Therefore, public policy surrounding these games is an open issue that merits further investigation.

As mentioned in Chakraborty's recent article, public policy refers to both governmental policy and non-government policy \cite{chakraborty2015}. Our project will examine the policies surrounding regulation of violent video games, including censorship and purchase restrictions. We will discuss all four aspects of regulation, law, architecture, market, and social norms, as each complicates the issue in a unique way.
\section*{Proposed References}
\nocite{*}
\bibliography{biblio}
\end{document}